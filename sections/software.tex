\providecommand{\main}{..}  
% *Modification: redefine path location, must go before \documentclass
\documentclass[../computer-history.tex]{subfiles}

\begin{document}
Only once digital computers started being introduced did calculators move from mechanical to electronic. Through this transition, the line between hardware and software greys as we move to smaller and smaller mechanisms. The first publicly available electronic calculator was ANITA, debuting in 1961. But with the use of vacuum tubes during a time when transistors were becoming the standard meant this calculator was doomed for a short shelf life.

This marked the downfall of mechanical calculators, and as competition rose, more and more companies dove in to make a profit. The distinction between calculator and computer became increasingly vague as calculators grew more and more sophisticated. The Mathatronics Marathon was the first programmable desktop calculator, lending an argument for simply buying a large calculator instead of a computer.

Shortly after, calculators got small. The first handheld four-function calculator was designed as early as 1967. The first handheld available to the general market sold for \$245 in 1971 and sold for \$10 within a decade. A year later in 1972, Hewlett-Packard released the HP-35 Pocket Calculator, sporting all the features of a fully scientific calculator in handheld form.

%********************************************************************************
%      Bibliography
%********************************************************************************
\biblio
\end{document}
