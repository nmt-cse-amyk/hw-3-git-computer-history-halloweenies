\providecommand{\main}{..}  
% *Modification: redefine path location, must go before \documentclass
\documentclass[../computer-history.tex]{subfiles}

\begin{document}
The hardware of the calculator started all the way back with the abacus. However in the 1940's, Sung Jin Pai, who was a Korean mathematician, created a way to replicate the abacus to human hands and was used to represent numbers from 1 to 99. The abacus was such an important piece of technology that it also ended up becoming the first pocket calculator.

The next piece of hardware seen in history for the calculator was the Antikythera mechanism. It is the oldest known scientific calculator. With an arrangement of over 30 gears it could determine the position of the sun, moon and planets, predict eclipses and even track the Olympic games' dates. This device has improved our understanding of the greeks' knowledge and technology.

This next piece of hardware could do a wide range of approximate calculations; such as basic arithmetic to even calculating volume and area. These piece of hardware was known as the Sector. Overall, there is a surplus of different hardware throughout history, even including items like The Pascaline. 



%********************************************************************************
%      Bibliography
%********************************************************************************
\biblio
\end{document}
